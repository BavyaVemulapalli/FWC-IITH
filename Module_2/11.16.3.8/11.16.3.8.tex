\documentclass{article}
\usepackage{amsmath,amssymb,amsfonts,amsthm}
\usepackage{enumitem}
\usepackage{hyperref,xcolor}
\hypersetup{
    colorlinks,
    urlcolor={black}  %black!50!blue
}
\newcommand{\Mod}[1]{\ (\mathrm{mod}\ #1)}
\let\vec\mathbf

\def\inputGnumericTable{}
\usepackage{array}
\usepackage{longtable}
\usepackage{calc}
\usepackage{multirow}
\usepackage{hhline}
\usepackage{ifthen}
\usepackage{mathtools}
\newcommand\Myperm[2][^n]{\prescript{#1\mkern-2.5mu}{}P_{#2}}
\newcommand\Mycomb[2][^n]{\prescript{#1\mkern-0.5mu}{}C_{#2}}
\providecommand{\cbrak}[1]{\ensuremath{\left\{#1\right\}}}
\newcommand{\Problem}{\noindent \textbf{Problem: }}
\newcommand{\solution}{\noindent \textbf{Solution: }}
\setlist[enumerate]{font=\small\bfseries}

\begin{document}
\title{\textbf{PROBABILITY 1}}
\author{Vemulapalli Bavya Sri - FWC22046}
\date{December 2022}


\maketitle

$\mathbf{11.16.3.8}$\\
    Three coins are tossed once Find the probability of getting (i) 3 heads  (ii) 2 heads   (iii) at least 2 heads   (iv) at most 2 heads  (v) no head  (vi) 3 tails   (vii) exactly two tails  (viii) no tail  (ix) at most two tails\\

\solution
When three coins are tossed once the sample space is given by
\vspace{0.1cm}
\begin{align}
   S=HHH,HHT,HTH,THH,HTT,THT,TTH,TTT 
\end{align}
\begin{align}
    n(s) = 8
\end{align}
(i) Let B be the event of the occurrence of 3 heads Accordingly 
\begin{align}
    B=HHH\\
    P(B) = \frac{n(B)}{n(s)} = \frac{1}{8}
\end{align}
(ii) Let C be the event of the occurrence of 2 heads Accordingly
\begin{align}
    C=HHT,HTH,THH\\
    P(C) = \frac{n(C)}{n(s)} = \frac{3}{8}
\end{align}
(iii) Let D be the event of the occurrence of at least 2 heads
Accordingly \\
\begin{align}
    D=HHH,HHT,HTH,THH\\
    P(D) = \frac{n(D)}{n(s)} = \frac{4}{8}
\end{align}
(iv) Let E be the event of the occurrence of at most 2 heads
Accordingly \\
\begin{align}
    E=HHT,HTH,THH,HTT,THT,TTH,TTT
\end{align}
\begin{align}
    P(E) = \frac{n(E)}{n(s)} = \frac{7}{8}
\end{align}
(v) Let F be the event of the occurrence of no head
Accordingly\\
\begin{align}
    F=TTT\\
    P(F) = \frac{n(F)}{n(s)} = \frac{1}{8}
\end{align}
(vi) Let G be the event of the occurrence of 3 tails
Accordingly\\
\begin{align}
    G=TTT\\
    P(G) = \frac{n(G)}{n(s)} = \frac{1}{8}
\end{align}
(vii) Let H be the event of the occurrence of exactly 2 tails
Accordingly \\
\begin{align}
    H=HTT,THT,TTH\\
    P(H) = \frac{n(H)}{n(s)} = \frac{3}{8}
\end{align}
(viii) Let I be the event of the occurrence of no tail
Accordingly\\
\begin{align}
     I=HHH\\
      P(I) = \frac{n(I)}{n(s)} = \frac{1}{8}
\end{align}
(ix) Let J be the event of the occurrence of at most 2 tails
Accordingly \\
\begin{align}
    J=HHH,HHT,HTH,THH,HTT,THT,TTH
\end{align}
\begin{align}
    P(J) = \frac{n(J)}{n(s)} = \frac{7}{8}
\end{align}
\end{document}